
\newglossaryentry{centrale inertielle}
{
	name=Centrale inertielle,
	description={Instrument utilisé en navigation, capable d'intégrer les mouvements d'un mobile (accélération et vitesse angulaire) pour estimer son orientation (angles de roulis, de tangage et de cap), sa vitesse linéaire et sa position.}
}
\newglossaryentry{timestamp}
{
	name=Timestamp,
	description={Enrengistrement de l'heure et/ou la date d'un événement.}
}

\newglossaryentry{traceur de G}
{
	name=Traceur de G,
	description={Indique et enregistre les forces d'accélération grâce à un capteur de pression numérique de haute précision intégré et à une plate-forme inertielle. }
}

\newglossaryentry{GPS}
{
	name=GPS,
	description={Le système de positionnement global (GPS) est un service public américain qui fournit aux utilisateurs des services de positionnement, de navigation et de synchronisation (PNT). Ce système se compose de trois segments : le segment spatial, le segment de contrôle et le segment utilisateur. L'U.S. Space Force développe, entretient et exploite les segments spatial et de contrôle.}
}

\newglossaryentry{GNSS}
{
	name=GNSS,
	description={Le système mondial de navigation par satellite (GNSS) est un terme général décrivant toute constellation de satellites qui fournit des services de positionnement, de navigation et de synchronisation (PNT) à l'échelle mondiale ou régionale.}
}

\newglossaryentry{PIC32}
{
	name=PIC32,
	description={Famille de microcontrôleur 32-bits de Microchip.}
}



\newabbreviation{mcu}{MCU}{microcontrôleur.}
\newabbreviation{imu}{IMU}{centrale inertielle.}
\newabbreviation{rf}{RF}{radio-fréquence.}
\newabbreviation{gps}{GPS}{global Positioning System.}
\newabbreviation{gnss}{GNSS}{global navigation satellite systems.}



